
% Template for Elsevier CRC journal article
% version 1.1 dated 16 March 2010

% This file (c) 2010 Elsevier Ltd.  Modifications may be freely made,
% provided the edited file is saved under a different name

% This file contains modifications for Procedia Computer Science
% but may easily be adapted to other journals

% Changes since version 1.0
% - elsarticle class option changed from 1p to 3p (to better reflect CRC layout)

%-----------------------------------------------------------------------------------

%% This template uses the elsarticle.cls document class and the extension package ecrc.sty
%% For full documentation on usage of elsarticle.cls, consult the documentation "elsdoc.pdf"
%% Further resources available at http://www.elsevier.com/latex

%-----------------------------------------------------------------------------------

%%%%%%%%%%%%%%%%%%%%%%%%%%%%%%%%%%%%%%%%%%%%%%
%%%%%%%%%%%%%%%%%%%%%%%%%%%%%%%%%%%%%%%%%%%%%%
%%                                          %%
%% Important note on usage                  %%
%% -----------------------                  %%
%% This file must be compiled with PDFLaTeX %%
%% Using standard LaTeX will not work!      %%
%%                                          %%
%%%%%%%%%%%%%%%%%%%%%%%%%%%%%%%%%%%%%%%%%%%%%%
%%%%%%%%%%%%%%%%%%%%%%%%%%%%%%%%%%%%%%%%%%%%%%

%% The '3p' and 'times' class options of elsarticle are used for Elsevier CRC
\documentclass[3p,times]{elsarticle}

%% The `ecrc' package must be called to make the CRC functionality available
\usepackage{ecrc}
\usepackage{setspace}
\usepackage{algcompatible}
\usepackage[usenames,dvipsnames]{color}
\usepackage{amsthm,amsmath}
\newtheorem{definition}{Definition}
\usepackage{graphicx}% Include figure files
\usepackage{dcolumn}% Align table columns on decimal point
\usepackage{bm}% bold math
\usepackage{hyperref}% add hypertext capabilities
\usepackage[mathlines]{lineno}

%% The ecrc package defines commands needed for running heads and logos.
%% For running heads, you can set the journal name, the volume, the starting page and the authors

%% set the volume if you know. Otherwise `00'
\volume{00}

%% set the starting page if not 1
\firstpage{1}

%% Give the name of the journal
\journalname{Social Networks}

%% Give the author list to appear in the running head
%% Example \runauth{C.V. Radhakrishnan et al.}
\runauth{}

%% The choice of journal logo is determined by the \jid and \jnltitlelogo commands.
%% A user-supplied logo with the name <\jid>logo.pdf will be inserted if present.
%% e.g. if \jid{yspmi} the system will look for a file yspmilogo.pdf
%% Otherwise the content of \jnltitlelogo will be set between horizontal lines as a default logo

%% Give the abbreviation of the Journal.
\jid{SN}

%% Give a short journal name for the dummy logo (if needed)
\jnltitlelogo{Social Networks}

%% Hereafter the template follows `elsarticle'.
%% For more details see the existing template files elsarticle-template-harv.tex and elsarticle-template-num.tex.

%% Elsevier CRC generally uses a numbered reference style
%% For this, the conventions of elsarticle-template-num.tex should be followed (included below)
%% If using BibTeX, use the style file elsarticle-num.bst

%% End of ecrc-specific commands
%%%%%%%%%%%%%%%%%%%%%%%%%%%%%%%%%%%%%%%%%%%%%%%%%%%%%%%%%%%%%%%%%%%%%%%%%%

%% The amssymb package provides various useful mathematical symbols
\usepackage{amssymb}
%% The amsthm package provides extended theorem environments
%% \usepackage{amsthm}

%% The lineno packages adds line numbers. Start line numbering with
%% \begin{linenumbers}, end it with \end{linenumbers}. Or switch it on
%% for the whole article with \linenumbers after \end{frontmatter}.
%% \usepackage{lineno}

%% natbib.sty is loaded by default. However, natbib options can be
%% provided with \biboptions{...} command. Following options are
%% valid:

%%   round  -  round parentheses are used (default)
%%   square -  square brackets are used   [option]
%%   curly  -  curly braces are used      {option}
%%   angle  -  angle brackets are used    <option>
%%   semicolon  -  multiple citations separated by semi-colon
%%   colon  - same as semicolon, an earlier confusion
%%   comma  -  separated by comma
%%   numbers-  selects numerical citations
%%   super  -  numerical citations as superscripts
%%   sort   -  sorts multiple citations according to order in ref. list
%%   sort&compress   -  like sort, but also compresses numerical citations
%%   compress - compresses without sorting
%%
%% \biboptions{comma,round}

% \biboptions{}

% if you have landscape tables
\usepackage[figuresright]{rotating}

% put your own definitions here:
%   \newcommand{\cZ}{\cal{Z}}
%   \newtheorem{def}{Definition}[section]
%   ...

% add words to TeX's hyphenation exception list
%\hyphenation{author another created financial paper re-commend-ed Post-Script}

% declarations for front matter

\begin{document}

\begin{frontmatter}

%% Title, authors and addresses

%% use the tnoteref command within \title for footnotes;
%% use the tnotetext command for the associated footnote;
%% use the fnref command within \author or \address for footnotes;
%% use the fntext command for the associated footnote;
%% use the corref command within \author for corresponding author footnotes;
%% use the cortext command for the associated footnote;
%% use the ead command for the email address,
%% and the form \ead[url] for the home page:
%%
%% \title{Title\tnoteref{label1}}
%% \tnotetext[label1]{}
%% \author{Name\corref{cor1}\fnref{label2}}
%% \ead{email address}
%% \ead[url]{home page}
%% \fntext[label2]{}
%% \cortext[cor1]{}
%% \address{Address\fnref{label3}}
%% \fntext[label3]{}

%\dochead{}
%% Use \dochead if there is an article header, e.g. \dochead{Short communication}

\title{Hierarchical Structure in Social Networks}

%% use optional labels to link authors explicitly to addresses:
%% \author[label1,label2]{<author name>}
%% \address[label1]{<address>}
%% \address[label2]{<address>}

\author[au1]{Cynthia Cook\footnote{Authors are listed in alphabetical order but all contributed equally to this publication.}} 
\author[au2]{Matthew J. Denny}
\author[au3]{Mitchell Goist} 
\author[au4]{Timmy Huynh}

\address[au1]{Department of Statistics cmc496@psu.edu}
\address[au2]{Department of Political Science, mzd5530@psu.edu}
\address[au1]{Department of Political Science mlg307@psu.edu}
\address[au2]{Department of Sociology and Criminology, tnh133@psu.edu}

\begin{abstract}

\end{abstract}

\begin{keyword}
Hierarchy \sep Network \sep Power

%% MSC codes here, in the form: \MSC code \sep code
%% or \MSC[2008] code \sep code (2000 is the default)

\end{keyword}

\end{frontmatter}

%%
%% Start line numbering here if you want
%%
% \linenumbers


\section{Introduction}
\label{sec:introduction}

\section{Measuring Hierarchy}

\subsection{Measuring Hierarchy in Groups}


\subsection{Measuring Hierarchy in Networks}


\section{A Model of Network Hierarchy}

\subsection{Measures of Hierarchy}
\begin{enumerate}
	\item Landau's $h\in[0,1]$ and Kendall's $K\in[0,1]$ are both used to compare a network to a perfect linear hierarchy in \cite{animals}, where $i=1...N$ is the number of nodes and $S_i$ is the row sum for each node also referred to as the dominance total: \\
	\begin{center}
		$h=\frac{12}{N^3-N}\sum_{i=1}^{N}{[S_i-\frac{N-1}{2}]}$, \\
		\vspace{5mm}
		IF we let $d$ be the number of cyclic triads defined as: $d=\frac{N(N-1)(2N-1)}{12}-\frac{1}{2}\sum{S_i^2}$. Then 
		\\ \vspace{5mm}
		$K=1-\frac{d}{d_{max}}$, where 
		
		$$
		d_{max} = \left\{ \begin{array}{rl}
		\frac{1}{24}(N^3-N)&\mbox{ if $N$ is odd} \\
		\frac{1}{24}(N^3-4N)&\mbox{ if $N$ is even}
		\end{array} \right.
		$$
	\end{center} 
	
	\item Triangle transitivity is shown to be higher in dominance relationships in \cite{animals}, but not used as a measure here.
	
	\item Treeness $T\in[-1,1]$ is the average of $f(G)$ over the set in $W(G)$, where $W(G)$ is the subset $G_C$ and all of its subsets obtained through a leaf removal algorithm. This measure is defined in \cite{3D}.
	$$
	f(G)=\frac{H_f(G_C)-H_b(G_C)}{max\{H_f(G_C),H_b(G_C)\}},
	$$
	where $H_f, H_b$ denote the forward and backward path entropies, respectively, where $h_f(v_i)=-\sum{P(\pi_k|v_i)logP(\pi_k|v_i)}$.  
	
	\item Feedforwardness $F\in[0,1]$ is the average of path weights $F(\pi_k)$ where cyclic modules that are closer to the top get a higher penalty, and defined by \cite{3D}. Here the paths under consideration are all paths starting at from the top of $G_C$ denoted $\pi_k$. Let $k=1,...M$ be this number of paths then:
	$$
	F(G)=\frac{1}{M}\sum_{k=1}^{M}{\frac{|v(\pi_k)|}{\sum{a_i}}},
	$$
	where $a_i$ are the weights of each node along the path under consideration (i.e. the number of collapsed nodes from $G$ in the corresponding node of $G_C$), and $v(\pi_k)$ is the number of nodes along the path $\pi_k$.
	
	\item Orderability $O\in[0,1]$ is the fraction of nodes that do not belong to any cycle and defined by \cite{3D}:
	$$
	O(G)=\frac{|v_i\in V_c\cap V|}{|V|}
	$$
	
	\item Global Reaching Centrality where the graph is unweighted and directed is defined by \cite{GRC} as:
	$$
	GRC=\frac{\sum_{i\in V}{[C_R^{max}-C_R(i)]}}{N-1},
	$$
	where $C_R(i)$ is the local reaching centrality defined as the proportion of all nodes in $G$ that can be reached along outgoing edges from node $i$.
	
	\item Global Reaching Centrality where the graph is weighted and directed is defined by \cite{GRC} using the following version for the reaching centrality:
	$$
	C_{R}^{'}(i)=\frac{1}{N-1}\sum_{j: 0<d^{out}_{(i,j)<\infty}}{( \frac{\sum_{k=1}^{d^{out}(i,j)} {w_{i}^{(k)} (j) } }{d^{out}(i,j)} )}
	$$
	
	\item Global Reaching Centrality where the graph is unweighted and undirected is defined by \cite{GRC} using the following version for the reaching centrality:
	$$
	C_{R}^{''}(i)=\frac{1}{N-1}\sum_{j:0<d(i,j)<\infty}{\frac{1}{d(i,j)}}
	$$
	
	\item Control Centrality in a weighted and directed graph, defined by \cite{Liu12}, identifies the minimum number of nodes need to drive an entire network to a given final state. Consider a directed, weighted network:
	$$
	\bm{x}(t) = \bm{Ax}(t) + \bm{Bu}(t)
	$$
	which is the state of each node at time t, and also denoted as \begin{math}(\bm{A},\bm{B})\end{math}. The components of this controllability matrix are: $\bm{A}\in \mathbb{R}^{N\times N}$, where each element gives the strength that node $j$ can affect node $i$; and $\bm{B}\in\mathbb{R}^{N\times M}$, where each element is the strength between the input signal $u_{j}(t)$ and node $i$, and M contains independent signals imposed by an outside controller. Defining $\bm{C} = (\bm{A}$,$\bm{B})$, the control centrality of node $i$ is:
	$$
	C_{c}(i) \equiv rank_{g}(\bm{C}^{i})
	$$
\end{enumerate}

\section{Data}

\begin{enumerate}

    \item Adolescent Health: survey asked students to list 5 male and female friends. \cite{AdHealth}
    
    \item Residence Hall: friendships between 217 students in Australian National University. \cite{Res}
    
    \item Taro Exchange: gift--giving relationships between households in a Papaun village. \cite{Taro}

    \item Highschool: friendship relationship between boys at a small Indiana high school in 1957-1958. \cite{HS}
    
    \item Dutch College: friendships between 32 university freshmen. \cite{Dutch}
    
    \item Monks: preference ratings between monks in a cloister during a crisis. \cite{monks}
    
    \item Physicians: innovation spread between 246 physicians in Illinois. \cite{docs}
    
    \item Seventh graders: activity specific proximity rankings for 29 middle school students in Victoria \cite{sevies}.
    
    \item Prosper loans: loans between users of prosper.com \cite{prosper}.
    
    \item Libimseti.cz: likes between users on a Czech dataing site \cite{libi}.
    
    \item Friendster: friendship adds on the online site Friendster \cite{friendster}.
    
    \item Digg: friendships on Digg \cite{digg}.
    
    \item Youtube: connections between Youtube users \cite{youtube}.
    
    \item Epinions: who--trusts--whom between users of epinions \cite{epinions}.
  
    \item EU emails: emails for 18 months from a major European research institution \cite{EU}.
    
    \item Facebook: friends lists from FAcebook, generated through a Facebook app survey \cite{facebook}.
    
    \item Google Plus: friends between users who selected to ``share circles'' on Google Plus \cite{facebook}.
    
    \item Linx kernel mailing list: communication network for the linux kernel mailing list, where each edge is a reply from a user to another \cite{linux}.

    \item Livejournal: map of an online community friendships of Livejournal users \cite{livejournal}.

    \item Manufacturing: communication network between employess of a mid--size manufacturing firm \cite{manufacturing}.
    
    \item Pokec: Friendship networks in the Pokec online social network, popular in Slovakia \cite{pokec}.
    
    \item Slashdot: tagging between users in slashdot for 2008 and 2009 \cite{livejournal}.
    
    \item Twitter: circles between twitter users \cite{facebook}.
    
    \item UC Irvine: messages sent between students on an online community at UC Irvine \cite{irvine}.
 
    \item U. Rovira i Virgili: email communication network from University Rovira i Virgili in Tarragona \cite{URV}.

    \item Wikipedia Talk: network of discussions between all users from the beginning of Wikipedia to January 2008 \cite{wiki}.

    \item Wikipedia Votes: data from administrator elections \cite{wiki}.
    
    \item Wikipedia Requests for Adminship: requests from 2003 through 2013 \cite{wiki2}.

    \item Friendster: network for online social site Friendster \cite{friendster}.

\end{enumerate}

\section{Analysis}



\section{Conclusions}

\bibliographystyle{elsarticle-num}
\bibliography{library}

%% Authors are advised to use a BibTeX database file for their reference list.
%% The provided style file elsarticle-num.bst formats references in the required Procedia style

%% For references without a BibTeX database:

% \begin{thebibliography}{00}
\begin{thebibliography}{9}
	\bibitem{animals} 
	Shizuka, D. and McDonald, D. B. (2012),
	\textit{A social network perspective on measurements of dominance hierarchies}. 
	Animal Behavior: 83, 925-934.
	
	\bibitem{3D} 
	Murtra, B., Goni, J., and Caso, C. (2013),
	\textit{On the origins of hierarchy in complex networks}. 
	PNAS: 110, 33, 13316-13321.

	\bibitem{GRC} 
	Mones, E., Vicsek, L., and Vicsek, T. (2012),
	\textit{Hierarchy Measure for Complex Networks}. 
	Plos ONE: 7, 3, 1-10.
	
	\bibitem{Liu12}
	Liu, Y., Slotine, J., and Barab{\'a}si, A (2012).
	\textit{Control centrality and hierarchical structure in complex networks}.
	Plos ONE: 7, 9.
	
	\bibitem{AdHealth}
	Moody, J. (2001).
	\textit{Peer influence groups: Identifying dense clusters in large networks}.
	Social Networks: 23, 4, 261-283.
	
	\bibitem{Res}
	Freeman, L., Webster, C., Kirke, D. (1998)
	\textit{Exploring social structure using dynamic three--dimensional color images.}
	Social Networks: 20, 2, 109-118.
	
	\bibitem{Taro}
	Schwimmer, E. (1973)
	\textit{Exchange in the Social Structure of the Orokaiva: Traditional and Emergent Ideologies in the Northern District of Papua}.
	St. Martin's Press.
	
	\bibitem{HS}
	Coleman, J. (1973)
	\textit{Introduction to mathematical sociology}.
	London Free Press Glencoe.
	
	\bibitem{Dutch}
	Van de Bunt, G., Van Deuijn, M., Snijders, T. (1999)
	\textit{Friendship networks through time: An actor-oriented dynamic statistical network model}.
	Computational and Mathematical Organization Theory: 5, 2, 167-192.
	
	\bibitem{monks}
	Breiger, R., Boorman, S., Arabie, P. (1975)
	\textit{An algorithm for clustering relational data with applications to social network analysis and comparison with multidimensional scaling}.
	Journal of Mathematical Psychology: 12, 3, 1975.
	
	\bibitem{docs}
	Coleman, J., Katz, E., Menzel, H. (1957)
	\textit{The diffusion of an innovation among physicians}.
	Sociometry: 253-270.
	
	\bibitem{sevies}
	Watts, D., Strogatz, S. (1998)
	\textit{Collective dynamics of `small world' networks}.
	Nature: 393, 1, 440-442
	
	\bibitem{cite}
	Prosper loans network dataset - KONECT, May 2015.
	
	\bibitem{libi}
	Brozovsky, L., Petricek, V. (2007)
	\textit{Recommender system for online dating service}.
	Proc. Znalosti: 29-40.
	
	\bibitem{friendster}
	Friendster network dataset - KONECT, May 2015.
	
	\bibitem{Digg}
	Hogg, T., Lerman, K. (2012)
	\textit{Social dynamics of Digg}.
	EPJ Data Science: 1, 5.
	
	\bibitem{youtube}
	Mislove, A., Marcon, M., Gummadi, K., Druschel, P., Bhattacharjee, B. (2007)
	\textit{Measurement and analysis of online social networks}.
	Proc. Internet Measurement Conference.
	
	\bibitem{epinions}
	Richardson, M., Agrawal, R., Domingos, P. (2003)
	\textit{Trust Management for the Semantic Web}.
	ISWC
	
	\bibitem{EU}
	Leskovec, J., Kleinber, J., Faloutsos, C. (2007)
	\textit{Graph Evolution: Densification and Shrinking Diameters}.
	ACM Transactions on Knowledge Discovery from Data (ACM TKDD): 1, 1.
	
	\bibitem{facebook}
	McAuley, J., Leskovec, J. (2012)
	\textit{Learning to Discover Social Circles in Ego Networks}.
	NIPS
	
	\bibitem{linux}
	Linux kernel mailing list replies network dataset - KONECT, May 2015.
	
	\bibitem{livejournal}
	Leskovec, J., Lang, K., Dasgupta, A., Mahoney, M. (2009)
	\textit{Community Structure in Large Networks: Natural Cluster Sizes and the Absence of Large Well-Defined Clusters}.
	Internet Mathematics: 6, 1, 29--123.
	
	\bibitem{manufacturing}
	Michalski, R., Palus, S., Kazienko, P. (2011)
	\textit{Matching organizational structure and social network extracted from email communication}.
	Lecture Notes in Business Information Processing: 87, 196-206.
	
	\bibitem{pokec}
	Takac, L., Zabovsky, M. (2012)
	\textit{Data Analysis in Public Social Networks}.
	International Scientific Conference \& International Workshop Present Day Trends of Innovations, Lomza, Poland.
	
	\bibitem{irvine}
	Opsahl, T., Panzarasa, P. (2009)
	\textit{Clustering in weighted networks}.
	Social Networks: 31, 2, 155-163.
	
	\bibitem{URV}
	Guimera, R., Danon, L., Diaz-Guilera, A., Giralt, F., Arenas, A. (2003)
	\textit{Self--similar community structure in a network of human interactions}.
	Phys. Rev. E.: 68, 6.
	
	\bibitem{wiki}
	Leskovec, J., Huttenlocher, D., Kleinberg, J. (2010)
	\textit{Predicting Positive and Negative Links in Online Social Networks}.
	WWWW
	
	\bibitem{wiki2}
	West, R., Paskov, H., Leskovec, J., Potts, C. (2014)
	\textit{Exploiting Social Network Structure for Person-to-Person Sentiment Analysis}.
	Transactions of the Association for Computational Linguistics: 2, 297-310.

    \bibitem{friendster}
    Friendster network dataset - KONECT, May 2015.
    
\end{thebibliography}
%% \bibitem must have the following form:
%%   \bibitem{key}...
%%

% \bibitem{}

% \end{thebibliography}

\end{document}

%%
%% End of file `ecrc-template.tex'. 
